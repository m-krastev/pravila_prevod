\chapter{Идиоми}
\section{Списък с основни идиоми}
\begin{description}
    \item[To (not) have a leg to stand on] - мога/ не мога да докажа нещо; имам/ нямам доказателства.\textit{ The problem is, if you don't have a witness, you don't have a leg to stand on. \dots He was sure there was a mole and when you catch them, you should have a leg to stand on. (Убеден беше, че има шпионин и когато го заловиш, трябва да имаш доказателства.)}
    \item[For good] - доста често не означава за добро, а завинаги
    \item[It does not make sense] - Странно, но и този израз се бърка доста често. Не означава "Не прави смисъл" или "Не се създава усещане", а Няма логика / Нещо не е така, както изглежда
    \item[We are in the same boat] - Не означава "С теб сме в една лодка", а че сме в една и съща ситуация; трябва да вършим нещата заедно, защото имаме един и същ проблем
    \item[You are barking up the wrong treе] – Не означава "Лаеш по грешното дърво", а се използва в ситуации, когато се опитваш да постигнеш нещо, но начинът, по който се опитваш да го направиш, е погрешен; или напълно грешиш или не си разбрал/схванал нещо
    \item[You can not have the best of both worlds] - Не означава "Не можеш да имаш най-доброто от двата свята", а не можеш да получиш всичко, което желаеш / трябва да направиш избор
    \item[A bad egg] - Не означава "Развалени яйца", а човек, на когото не може да се вярва или се държи нечестно. Съвсем спокойно в зависимост от контекста може да се използват и типично нашенските изрази "От кофти тесто съм замесен" или "От лоша семка съм"
    \item[Try walking in my shoes] - Определено не означава "Опитай се да ходиш с моите обувки", а Опитай се да се поставиш на мое място
    \item[Know-it-all] - всезнайко
    \item[to be/fall head over heels] - влюбен съм
    \item[to cost an arm and a leg] - имам много висока цена; струвам адски скъпо
    \item[hit the road] - не е ритвам пътя, а потеглям, тръгвам на път, хващам пътя
    \item[take a flight] - 1. тръгвам си, отивам си; 2. тръгвам /отпътувам със самолет
    \item[in cold blood] - действие извършено с изключително хладнокръвие и жестокост
    \item[under the weather] - не се чувствам добре
    \item[I am so blue] - Не означава "Толкова съм син", а Тъжно ми е
    to drop (someone) a linе – пиша /изпращам бележка или писмо на някого
    \item[enough is enough] - стига толкова, изчерпи ми се търпението
    \item[Back/bet on the wrong horse] - Да поддържаш/ защитаваш неподходящия човек, макар че дори и буквално преведено на български смисълът не се губи.
    \item[I am in two minds] - Двоумя се / не мога да взема решение
    \item[I am all ears] - Слушам нещо с интерес/Целият съм в слух
    \item[Alter ego] - много близък прятел. Буквалният превод на латинската фраза е «другото ми Аз»
    \item[Beating a dead horse] - не означава да биеш умрял кон mosking.\item[gif] - изразът се използва когато някой се опитва да предизвика интерес и внимание, но без никакъв успех
    \item[Big cheese] - не е голямо сирене, а се използва за шеф, подобно на нашето "голяма клечка"
    \item[Bite the bullet] - "не е захапи куршума" или "гризни дървото", а да се изправиш пред нещо неприятно, което не можеш да избегнеш
    \item[Break a leg] - определено не е "счупи си крак", а се използва като пожелание за успех, късмет
    \item[Cold feet] - не е "студени крака", а по-скоро еквивалент на нашето "разтреперват ми се мартинките", т.е. плашиш се, губиш кураж да свъриш нещо
    \item[Cold turkey] - ако го срещнете във филм, в който са намесени наркозависими, определено не означава студено пуешко, а "внезапно прекъсване на приема на наркотици"
    \item[Cuckoo in the nest] - не е "кукувица в гнездо" , а проблем, който ескалира бързо
    \item[Dead air] - не е умрял/мъртъв въздух, а пълна тишина
    \item[Dead man walking] - не е зобми или ходещ мъртвец, а човек, попаднал в беда или опасна ситуация, на път е да загуби работата си, обществени позиции, семейството си и т.н.
    \item[Devil's advocate] - не е адвокат на Дявола, а някой, който спори, извърта аргументите и защитава позиции, в които не вярва, но го прави от любов към спора. Може да се използва и нашият израз "чете Евангелието като Дявола"
    \item[Dog days] - не означава "кучешки дни", а много горещи летни дни
    \item[Dog eat dog] - съперничество, конкуренция
    \item[Don't push my buttons!] - Не ме нервирай/дразни
    \item[Donkey's years] - не е магарешки години, а дълъг период от време
    \item[Piеce of cakе] – не е парче торта, както често го виждам преведено, а фасулска работа, лесна работа
    \item[Jackass] - не е задник, а по-скоро тъпанар, идиот, глупак и т.н.
    \item[badass] - има много значения, но определено не е "злогъз". Най-често се използва за самоуверен и силен мъж тип "мачо"
    \item[Fast and furious] - понякого може да изначава "бързи и яростни", но като идиом се използва в ситуации, когато събитията се развиват мълниеносно
    \item[Fishy] - не е рибешко, а по-скоро е еквивалент на българското "Работата мирише", т.е. има нещо подозрително
    \item[Fly on the wall] - като идиом не означава "муха на стената", а се използва за някой, който е видял или чул нещо, т.е. свидетел на събитието
    \item[Get your feet wet] - придобивам опит, първи стъпки в някакво начинание
    \item[Give me a hand] - един друг израз, който се бърка много често и се превежда като "подай ми ръка", а в действителност означава помогни ми
    \item[Go bananas] - не е "върви за банани", а изразява душевно състояние на силно вълнение, тревога или безпокойство
    \item[Go fry an egg] - не е отиди да си изпържиш яйце, а разкарай се, остави ме на мира
    \item[Gone fishing] - не е отиде за риба, а отново изразява душевно състояние на обърканост, човек, който не е наясно какво се случва около него
    \item[Hair of the dog] - не е нито кучешка козина, нито косата на кучето, а чисто и просто махмурлук
    \item[Have a go] - друг израз, който често се превежда като "Върви" или "Имаш разрешение",
    а всъщност означава да се опиташ да направиш нещо, дори и да мислиш, че нямаш големи изгледи за успех , т.е. пробвай се
    \item[Over your head] - Заел си се с нещо, което е извън възможностите ти
    \item[have the guts/balls] - няма нищо общо с черва и други атрибути 3.gif , а е стиска ми, имам кураж да направя нещо
    \item[Kick ass] - жестоко, върховно
    \item[Put the moves on someone] - опитвам се да прелъстя, свалям
    \item[Way to go] - браво
    \item[in the middle of nowhere] - затруднено положение, безизходица, някъде далеч (на края на света)
    \item[Over your head] - Заел си се с нещо, което е извън възможностите ти
    to be/\item[sit on the fence] - изчаквам, колебая се, чакам да видя накъде ще духне вятъра
    \item[rock the boat] - създавам смут
    \item[Low man on the totem pole] - човек току-що постъпил на работа - новак в службата или най-ниско в служебната йерархия
    \item[tire kicker] - термин използван от дилърите на коли за човек,
    който постоянно се отбива в салоните за коли, оглежда колите, "подритва гумите", но никога не купува
    \item[Two-A-days] - или още познат като "Hell week", e футболен термин, който се използва, когато един футболен отбор има по две тренировки на ден
    \item[break my balls] - ядосвам, дразня, подигравам се с някого, правя номер на някого
    \item[To beat around the bush] - говоря с недомлъвки, увъртам
    (to be)\item[ like a bull in a China shop] - на български може да се преведе като "слон в стъкларски магазин"
    \item[The writing on the wall] - предзнаменование, поличба, знамение за нечия съдба, участ, орис или нещастие.
    \item[to all intents and purposes] - предимно или всъщност. В зависимост от контекста може да се преведе и като както и да го погледнем.
    \item[Dressed to kill] - спокойно, човекът никого няма да убива, той е просто много добре облечен.
    \item[Good shit] - Добра работа, добре свършено, браво!
    \item[Аt the end of the day] - В крайна сметка, в края на краищата
    \item[Pulling 'someone's leg] - подиграваш се с някой, правиш си бъзик
    \item[state of the art] - най-доброто средство; последна дума в техниката или технологията; ултрамодерно
    \item[to call a spade a spade] - назовавам нещата с истинските им имена
    \item[to come off cheap] - отървавам се леко
    \item[to lay by for a rainy day] - бели пари за черни дни
    \item[by hook or by crook] - на всяка цена
    \item[blood is thicker than water] - кръвта вода не става
    \item[as ugly as sin] - грозен като смъртта
    it never rains,\item[but it pours] - злото никога не идва само
    \item[to go through thick and thin] - минавам през огън и лед
    \item[I have to go/I got to go] - освен че значи трабва да вървя, може да се използва в хиляди комбинации като "трябва да свърша работата", "ходи ми се до тоалетната" и т.н
    \item[Goodbye for good] - сбогом завинаги
    \item[I'll/I will take it from here] - не е буквално аз ще го взема от тук, а аз поемам от тук.
    \item[just married] - младоженци
    \item[team] - комбина; често се превежда неправилно като отбор.
    \item[To ride shotgun] - возя се отпред, на седалката до шофьора.
    \item[Call shotgun] - заплювам си да се возя отпред ( ако е в този контекст )
    Out with the old, in with the new означава - каквото било - било; да забравим миналото (и да гледаме в бъдещето)
    \item[To be in the zone] - не означава, че някой се "намира в зоната", а означава, че е във върховата си форма, в стихията си, в силата си или в победна серия.
    \item[To drop a bomb on someone] - не означава, че "мятате бомба някому отгоре", а означава, че го шокирате.
    \item[Go postal] - да се вбесиш, да се ядосаш
    \item[to be on one's ear] - пиян съм
    \item[to be on one's second wind] - отпочинал съм
    \item[blue twoes] - полицейски автомобил
    \item[peeler] - полицай
    \item[to have more money than cents] - пилея пари, харча пари неразумно
    \item[to be away in the head] - върша необмислени неща
    \item[to be the gipsy in the house] - 1. неудачник 2. човек, който не се облича добре
    \item[SKELETON in the cupboard или family SKELETON] - неприятна/позорна семейна тайна.
    \item[SKELETONS in the closet] - тайна от миналото на някого, която е на път да бъде разкрита.
    \item[born with a SILVER SPOON in one's mouth] - роден с късмет, предопределен да бъде богат.
    \item[to SHOOT the bull] - хваля се, преувеличавам.
    \item[there will be the DEUCE to pay] - ще ти излезе coлено/през носа.
    \item[to ring a BELL] - разг. извиквам спомени, сещам се, спомням си.
    \item[God gives short horns to a cursed cow] - На бодлива крава Господ рога не дава.
    \item[German virgin] - Две девятки при игра на покер(Texas hold 'em).
    \item[my SHIP comes home] - забогатявам.
    \item[my SHIP has sailed] - разминавам се с богатството.
    \item[BLOWER] - телефон, тех. компресор.
    \item[out on a LIMB] - неизгодно/опасно положение.
    \item[gone to his REWARD] - починал, отишъл си от тоя свят, на оня свят
    \item[Elbowing someone out of the way] - Изблъсквайки с лакти някой, който ти се е изпречил на пътя
    \item[My knight in shining armour] - моят принц или моят приказен принц
    \item[Five-by-five / Five by five / 5 by 5 / 5-by-5 /] - Разбирам те напълно добре.
    Take the Michael /\item[ take the Mickey] - бъзикам се с някого
    \item[Bob's your uncle] - И всичко ще е точно
    \item[Get two bites at the cherry] - Постъпвам, действам неуверено
    \item[To hit the sack] - отивам да си лягам, гушвам възглавницата
    \item[to be up for grabs] - ставам обществено достояние, съм общодостъпен, за всички
    \item[Back to porridge] - обратно към всекидневието
    \item[to put my foot in my mouth] - казвам нещо обидно или глупаво; казвам нещо, за което после съжалявам
    \item[to bring/let/take someone down a peg or two] - скастрям някого, поставям някого на мястото му, смачквам фасона на някого
    \item[And what have you / And what you have] - и прочие.
    \item[Second to none] - на висота; по-добро от всичко.
    \item[You are avocado in my chop salad] - Специален си (специална си) за мен
    Keep (one's)\item[ head] - запазвам самообладание
    Clean (one's)\item[ clock] - побеждавам убедително
    \item[My teeth are floating] - много ми се ходи по малка нужда
    \item[few sandwiches short of a picnic] - луд или откачен. Демек шашав по цялата глава.
    \item[one card shy of a full deck] - същото като горното.
    \item[to go banana] - полудявам, откачам.
    \item[to pull the trigger] - освен дословно, където е необходимо, се превежда и като "върша нещо с ясното съзнание, че може да завърши зле за мен" или чисто и просто "действам решително"
    \item[like a bat out of hell] - правя нещо много бързо.
    \item[road hard and put away wet] - Нещо е използвано много и без да се полагат грижи за него. Лафът идва от язденето на коне. Преуморил си коня от яздене, та чак се е изпотил и после не си го сресал и попил, преди да го прибереш в конюшнята.
    \item[What's for tea?/What do you want for tea?] - Какво има за вечеря?/Какво искаш за вечеря?
    \item[slam dunk] - супер забивка
    \item[call the shots] - контролирам нещата
    \item[kick a habit] - спирам да върша нещо, което преди ми е било навик
    \item[kick back] - не значи ритам назад, а значи - да си лежа и да не правя нищо.
    \item[Get my rocks off/Got his rocks off] - Правя си кефа веднъж, той си направи кефа веднъж
    \item[to play it by ear/play something by ear] - отгатвам по нюх, по интуиция
    Every now and then/\item[Every once in a while] - от време на време, понякога
    \item[to go all the way] - стигам до края (в секса)
    \item[taking speed] - взимане на амфетамини
    to have/\item[get someone by the short hairs] - държа някого във властта си, водя някого за носа
    \item[to be on pins and needles] - на тръни съм, изправен на нокти, седя като на тръни
    \item[if/when push comes to shove] - ако/като ножът опре до кокала
    \item[You are quite dry] - Доста си безчувствен, неемоционален, безпристрастен, хладнокръвен, ироничен, жаден.
    When it rains,\item[ it pours] - От трън, та на глог
    \item[Drive up the wall] - да дразниш, да ядосваш някого
    \item[to go for wool and come home shorn] - връщам се с подвита опашка или готината приказка "отивам като аслан, връщам се посран"
    \item[Here's to you / here's how] - Наздраве!; За ваше здраве!
    \item[Here's to Mary!] – За Мери!
    \item[Here's to our victory!] – За победата (ни)!, а не Тук е победата ни!
    \item[it is Greek to me/you] - за мен/теб това е напълно неразбираемо; за мен това е като (написано) на китайски, звучи ми като на китайски
    \item[there is no use crying over spilt milk] - роненето на сълзи с нищо не помага; не може да върнеш времето назад; няма да върнеш изгубеното
    \item[a babe in arms] - като малко дете
    a babe in the wood(s) – доверчив, неопитен, некадърен, непрактичен човек; като малко дете
    \item[crow's feet] - пачи крак; бръчки в крайчеца на очите, а не гарванови крака, както го видях веднъж.
    \item[the shit hit the fan] - започват големи неприятности
    
\end{description}

\section{Цветни идиоми}
Тясната връзка между цветовете и емоциите намира колоритен израз - и буквално, и преносно, в образната реч на идиомите, в които участват цветове. По-долу са дадени най-употребяваните изрази, обагрени в бяло, черно, червено, розово и жълто; а във втората част ще разгледаме сини, зелени, кафяви, сиви, сребристи и златисти идиоми, както и такива, без определен цвят. 

\subsection{Бели идиоми}
В британската култура, както и у нас, белият цвят символизира чистота и невинност.

\begin{description}
    \item[Flying Head/Mind] - Това определено в преносен смисъл би трябвало да значи: Отнесен,мечтател, фантазьор и другите такива синоними...
    \item[tell a white lie] - (изричам) невинна, безобидна лъжа;
    \item[We take(step) on the wrong foot] - Буквално съм го виждал този израз как ли не преведен, но принципно е: Не почнахме като хората или сгрешихме в началото ... 99\% става дума за отношения между хората
    \item[as white as a sheet] - (ставам) блед като платно;
    \item[as white as a ghost] - смъртно бледен от страх/болест/шок;
    \item[white elephant] - ненужна, безполезна (често със скъпа поддръжка) собственост/придобивка;
    \item[a white-collar worker] - 1.служител в офис (ср. blue-\item[collar] - работник във фабрика, човек ,който се занимава с физ. труд). 2. ръководство, управление
    \item[great white hope] - надежден спасител, спасител от беда, спасител от големи проблеми
    \item[white coffee] - кафе с мляко (заб. но не и white tea; вместо това - \textbf{tea with milk})
    \item[whiter than white] - невинен като ангел; чист като сълза, в 1смисъл честен, искрен;
    \item[white sale] - традиционна разпродажба на чаршафи, кърпи, покривки и пр. на намалени цени.
\end{description}

\subsection{Черни идиоми}
Черният цвят традиционо свързваме със смъртта, вещерството, злите сили, празнотата, а след Черния четвъртък през 1929 г. този цвят придоби значение и на загуба, застой, отчаяние.

\begin{description}
    \item[in the black] - 1.на печалба; 2. печеливш, успешен (подход, идея, фирма);
    \item[black eye] - насинено (от удар) око;
    \item[black look] - свиреп поглед;
    \item[black and blue] - със синини (от удари) по тялото;
    \item[pot calling (calls) the kettle black] - присмял се хърбел на щърбел;
    \item[black out] - 1. затъмнявам (като изгасям осветлението или покривам прозорците); също - информационно затъмнение; 2. губя съзнание, припадам 3. цензурирам;
    \item[in black and white] - 1.черно на бяло, написано 2. ясен, ясно разбираем;
    \item[black-and-white] - черно-бял (и прен.); като кон с капаци;

\end{description}

\subsection{Червени идиоми}
Най-емоционално зареденият цвят – червеното, има и най-богата символика; свързва се с гняв, страст, любов, отмъщение, огън, здраве, война, жестокост, опасност и пр.; за червенокосите хора се говори, че са с огнен темперамент и силен дух.

\begin{description}
    
\item[as red as a cherry = as red as a poppy] - ярко червен;
\item[as red as a ruby = as red as blood] - тъмно червен;
\item[in the red] - на червено, на загуба (ср. in the black – на печалба);
\item[a redneck] - 1. невежа; селяндур. 2. „мутра”;
\item[a red flag] - червена лампа, в см. сигнал за опасност/нередност;
\item[red-eye flight] - нощен полет (когато самолетът излита късно през нощта и каца много рано сутринта);
\item[to be shown the red card] - уволнен съм;
to catch someone red-\item[handed] - хващам някого на местопрестъплението;
\item[to see red] - обезумявам от ярост, пада ми пердето;
\item[red tape] - бюрокрация (също - забавяне поради тромави административни процедури);
\item[roll out the red carpet] - посрещам някого с големи почести, устройвам салтанати;
\item[paint the (old) town red] - гуляя, забавлявам се по барове и кръчми; удрям го на живот;
\item[a red-letter day] - празничен, неработен ден (отбелязан на календара в червено);
\item[to see the red light] - надушвам приближаваща опасност, светва ми червената лампичка;
\item[red-light district] - район с червени фенери, т.е. където има проституция;
\item[redcap (ам)] – носач на гара или летище;
\item[red herring] - нещо, което отвлича вниманието от основния проблем; подвеждаща информация.
\item[like waving a red flag in front of a bull] - върша нещо, което със сигурност ще вбеси някого; играя си с огъня;
\end{description}

\subsection{Розови идиоми}
Розовото е момичешкият, "сладкият" цвят; символизира радост, оптимизъм, красиви мечти.

\begin{description}
    \item[pink slip] - предизвестие, дадено от работодателя, за прекратяване на трудов договор;
    \item[see pink elephants] - фантазирам, въобразявам си;
    \item[pink-collar workers] - офис служителки;
    \item[in the pink] - в отлично здраве;
    tickled pink –страшно доволен, кефлия.
\end{description}


\subsection{Жълти идиоми}
Жълтият цвят има двузначна символика; от една страна той се асоциира със светлината, вярата, божественото, а от друга – с малодушието, предателството, завистта и злобата.
\begin{description}
    \item[a yellow streak (to have a yellow streak down one's back )] – малодушие (като черта на характера) yellow-bellied (разг., остар.) - бъзлив, страхлив;
    \item[yellow-livered] - страхлив, малодушен;
    \item[be yellow] - страхопъзльо съм;
    \item[yellow flag] - карантина;
    \item[yellow press/journalism (ам)] – жълта преса / журналистика;
    \item[yellow jacket] - 1.сънотворно лекарство (от : барбитурата Нембутал, продаван през 70-те год. в жълти капсули); 2.някои видове дрога.
    \item[jack-of-all-trades] - широкоскроен човек; който го бива за всичко;
    \item[gardenvariety] - обикновен, типичен;
    \item[a clean bill of health] - добро здравословно състояние; добра диагноза от лекар;
    \item[to keep out of someone's hair] - не създавам проблеми на някого, не притеснявам някого, не го дразня.
    \item[taken for a ride] - прекарвам някого; "прецаквам" някого
    \item[take a pot luck] - рискувам
    \item[give somebody your backing] - помагам на някого
    \item[as dead as a dodo] - нещо без бъдеще, почти невъзможно
    \item[like a lamb to the slaughter] - букв. като агне за клане; покорно
    \item[he/she wouldn't hurt a fly] - букв. не би наранил и муха; невинен
    \item[fish out of water] - букв. като риба на сухо; в безизходица; не на място
    \item[chicken out] - плаша се
    \item[bee line] - най-краткият път до нещо; възможно най-бързо
    \item[walk like a cat on a hot tin roof] - ходи несигурно; нестабилно 
\end{description}


\subsection{Сини идиоми}
Синьото символизира интелект, мъдрост, духовно начало,
вярност, постоянство, искрена любов, непорочност, вечност, спокойствие.

\begin{description}
    \item[to blue pencil something] - цензурирам, налагам цензура;
\item[a blue-eyed boy] - 1. протеже; фаворит, млад мъж, ползващ се с подкрепата
на висшестоящ и лансиран заради специални заслуги. 2. млад мъж, приветстван (неохотно) за успехите, които е постигнал;
\item[to look / feel blue] - изглеждам/чувствам се тъжен;
\item[blue in the face] - посинял от яд; силно развълнуван (до пръсване);
\item[once in a blue moon] - от дъжд на вятър;
\item[men/boys in blue] - полицаи (униформени);
\item[out of the blue = out of a clear blue sky] - изневиделица, като гръм от ясно небе;
\item[blue funk (to be in a ~)] – отчаян до смърт, (изпадам) в черна дупка;
\item[blue ribbon] - първа награда; екстра качество, „супер”;
\item[talk a blue streak] - не ми млъква устата, говоря много и бързо;
\item[a blue movie] - филм за възрастни (с еротични сцени);
\item[scream blue murder] - оревавам орталъка; вдигам врява; протестирам шумно. 
\end{description}


\subsection{Зелени идиоми}
Зеленият цвят има дуалистичната природа – свързваме го както с живота, растежа, плодородието, така и със завистта, злобата, измамата.

\begin{description}
    \item[to be green] - незрял; новак; неопитен; аджамия;
\item[give someone the green light] - давам позволение, разрешавам (някому да направи/започне нещо);
\item[grass is always greener on the other side] - чуждата кокошка патка изглежда;
\item[green belt] - зелени площи край града;
\item[green power] - разг. престиж и власт, произтичащи от пари;
\item[green-eyed monster] - ревност;
\item[green with envy] - умирам, пръскам се от завист;
\item[green thumb] - талант за градинарство;
\item[green around the gills] - жълт-зелен (от страх/болест);
\end{description}

\subsection{Кафяви идиоми}
Кафявото свързваме със Земята, есента, меланхолията, покоя, консервативността.

\begin{description}
     \item[to be browned off] - отегчен (съм) до смърт; до гуша ми е дошло;
    \item[brown out] - смущения в електрозахранването, когато токът ту спира, ту идва;
    \item[brown paper bag] - (жарг. на радиоводещите в градовете) – полицейска кола без обозначения;
    \item[as brown as a berry] - черен като циганин (от слънце);
\end{description}

\subsection{Сребристи идиоми}
Среброто (като цвят) е символ на Луната, женското начало, непорочността, чистотата.

\begin{description}
    \item[the silver screen] - киното (като изкуство);
\item[every cloud has a silver lining] - всяко зло за добро.
\item[brown study] - замислено, вглъбено състояние/настроение;
\item[to brown-nose] - подмазвам се, „четкам”.
\end{description}


\subsection{Златисти идиоми}
Златото (като цвят) е символ на Слънцето, мъжкото начало, духовното извисяване, божественото, силата, безсмъртието.

\begin{description}
     \item[a golden opportunity] - златна възможност, голям шанс;
    \item[a golden handshake] - парично обезщетение при пенсиониране, най-вече на служител с висок пост;
    \item[a golden boy] - златно момче, най-вече в спорта - за изключително надарен и можещ състезател.
\end{description}

\subsection{Неоцветени идиоми}

\begin{description}
    \item[a horse of another/a different colour (ам)]- (това е) друг въпрос, нещо различно;
\item[to change colour] - пребледнявам;
\item[give colour to] - потвърждавам, придавам достоверност;
\item[(to be) colourless] - безличен (съм), незабележим (за човек, който няма ярка индивидуалност);
\item[off-colour joke] - мръсен виц;
\item[with flying colours] - с голям успех;
\item[to paint in bright/dark colours] - представям, обрисувам нещо в ярки/мрачни краски;
\item[to show oneself in one's true colours] - разкривам истинската си същност;
\item[to see someone in his true colours] - виждам истинското лице на някого (едва сега разбирам що за човек е);
\item[a highly coloured report] - доклад, който съдържа силно преувеличеили предубедена информация. 
\end{description}

\section{Британски изрази и идиоми}

\begin{description}
    \item[Bees Knees] - не означава "коленете на пчелите", а невероятен, приказен, фантастичен.
    \item[Wanker;Tosser] - означават едно и също нещо:онанист или по-просто казано чек***ия.
    \item[On your bike] - по-учтива форма на "fuck off"
    \item[Nancy boy] - изнежен мъж, но също така се употребява и за гей
    \item[Brassed off] - дошло ти е до гуша
    \item[Give us a bell] - обади ми се
    \item[On the job] - означава работя усърдно, но също така,правя секс
    \item[Total pants] - пълен боклук,обикновено се употребява за предмети,не за хора
    \item[Reverse the charges] - да звъниш на някого по телефона за негова сметка

\end{description}

\section{Други изрази}
\href{https://www.usingenglish.com/reference/idioms/}{Голяма база с идиоми може да намерите на този линк.}