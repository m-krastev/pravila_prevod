\chapter{Военни термини}

\section{Военни звания}
\begin{longtable}{|p{0.25\textwidth}|p{0.25\textwidth}|p{0.25\textwidth}|p{0.25\textwidth}|}
\hline
\textbf{US Army} & \textbf{Български аналог СВ и ВВС} & \textbf{US Navy} & \textbf{Български аналог ВМС} \\
\hline
\endhead
\hline
\endfoot
\hline
\caption{Военни звания} \label{tab:voenni-zvanija} \\
\endlastfoot
\hline
 Private & Редник & Seaman & Матрос
\\ Corporal & Ефрейтор & Petty Officer Third Class & Старши матрос
\\ Sergeant & Сержант & Petty Officer First Class & Старшина І степен
\\ First Sergeant & Старши сержант & Senior Chief Petty Officer  &Главен старшина
\\ Sergeant Major & Старшина & Master Chief Petty Officer & Мичман
\\ Warrant Officer & Офицерски кандидат && Офицерски кандидат  
\\ Second Lieutenant & Младши лейтенант & Sub Lieutenant & Младши лейтенант
\\ Lieutenant & Лейтенант & Ensign & Лейтенант
\\ First Lieutenant & Старши лейтенант & Lieutenant Junior Grade & Старши лейтенант
\\ Captain & Капитан & Lieutenant & Капитан-лейтенант
\\ Major & Майор & Lieutenant Commander & Капитан ІІІ ранг
\\ Lieutenant Colonel & Подполковник & Commander & Капитан ІІ ранг
\\ Colonel & Полковник & Captain & Капитан І ранг
\\ Brigadier General & Бригаден генерал & Commodore (Lower Half) & Комодор
\\ Major General & Генерал-майор & Rear Admiral (Upper Half) & Контраадмирал
\\ Lieutenant General & Генерал-лейтенант & Vice Admiral  &Вицеадмирал
\\ General & Генерал & Fleet Admiral & Адмирал
\\
\end{longtable}

Други:
\begin{itemize}
    \item AAM = Lieutenant Colonel and above /от подполковник и нагоре/
    \item ARCOM = Colonel and above /от полковник и нагоре/
    \item MSM = Major General and above /от генерал-майор и нагоре/
    \item LM = Lieutenant General and above /от генерал-лейтенант и нагоре/
    \item DSM = U.S. Army Chief of Staff /началник-щаб на американската армия/
    \item AWOL = Absent Without Official Leave - напуснал без разрешение
    \item POW = prisoner of war - именно военнопленник, а не военнозатворник, затворник от войната и подобни.
\end{itemize}

\section{Военни термини и фрази}

\begin{longtable}{|p{0.3\textwidth}|p{0.7\textwidth}|}
\hline
\textbf{Фраза} & \textbf{Превод} \\
\hline
\endhead
\hline
\endfoot
\hline
\caption{Често употребявани фрази} \label{tab:frazi} \\
\endlastfoot
To read & слушам радиопредаване (качество на приемане, останало в езика на радиотелеграфистите от времето на азбуката на Морз)\\ 
"Do you read me" & Как ме чуваш?\\ 
"Read you loud and clear" & Чувам те силно и ясно.\\ 
Copy & "Разбрах". "Копирането" идва от времената, когато радистите записвали съобщението на хартия\\ 

Roger & "Разбрах ви" или "Разбрано". Не особено уставен отговор. Означава, че предадената информация е приета и разбрана. НЕ означава съгласие за изпълнение (молба или заповед). От първата буква на думата "Received"("получено").\\ 

Acknowledged & "Прието". Подтвърждение за приемане на предаването. По-дълъг аналог на "Roger".\\ 

Wilco & "Ще бъде изпълнено", "Слушам". Заповедта е разбрана и ще бъде изпълнена. Съкращение от Will comply.\\ 

Over & "Приемам". В случай не означава "Край" /End/\\ 
Over and out & "Край на връзката".\\ 
"Zipper" & Потвърди получаването на радиосъобщението с две кратки натискания на бутона на микрофона.\\ 
"There are Indians on your way" & "Имаш противник пред себе си" /на известно разстояние/\dots Останало съобщение от времето на Дивия Запад, когато индианците са били враг на "демократите".\\ 
\end{longtable}

\begin{longtable}
{|p{0.3\textwidth}|p{0.7\textwidth}|}
\hline
\textbf{Фраза} & \textbf{Превод} \\
\hline
\endhead
\hline
\endfoot
\hline
\caption{Военен радиообмен} \label{tab:callouts} \\
\endlastfoot
\hline

Tally & "Виждам целта!".\\
No joy & "Целта не се вижда!" (именно "не се вижда", а не "Тук няма никаква цел". Усетете разликата!)\\

Bandit & Вражески самолет/въртолет.\\

Bogie & Самолет/вертолет с неизвестна принадлежност.\\

Visual & "Виждам". Визуален контакт със своя самолет/въртолет.\\

Blind & "Не виждам". Отсъствие на визуален контакт със свой самолет/въртолет.\\

Winchester & "Празен съм". Предаващият е изразходвал боекомплекта си.\\

Go wet/dry & "Аз съм мокър/сух". Самолет/въртолет пресича бреговата линия в посока на морето/сушата.\\

Sunrise & "Изгрев". Сигнал на екипажа, че започва да получава данни от външни източници (AWACS или наземен радар).\\

Midnight & "Полунощ". Сигнал на екипажа, че спира да получава данни от външни източници (AWACS или наземен радар).\\

Tumblweed & "Нищо не се вижда". Отсъствие на визуален/радарен контакт с който и да е. Запитване за допълнителна информация.\\

Splash & "Целта е унищожена" (за самолет); "Пряко попадение" (за удар по наземна цели).\\

Fox one & Изстрелване на ракета с радиолокационно самонасочване.\\

Fox two & Изстрелване на ракета с инфрачервено /топлинно/ самонасочване.\\

Fox three & Изстрелване на ракета с увеличен обсег на действие AMRAAM или Phoenix.\\

Fox four & Стрелец от бомбардировач имитира стрелба по цел.\\

Fox mike & FM радио\\

Hotel fox & HF радио\\

Uniform & UHF/AM радио\\

Victor & VHF/AM радио\\

Angels & Височина в хиляди футове /трябва да се превърне в съответните метри /приблизително/.\\

Bingo & Сигнал, че в самолета/въртолета е останало само гориво според първоначалния план за действие и завръщане /без да се посяга на резерва/.\\

Joker & Сигнал (след Bingo), че в самолета/въртолета е останала гориво на краен минимум /връщане с използване на резервата/ - както му казваме "да се върне на бензинови пaри"/.\\

Bullseye & Условна точка на местността, от която се отчитат относителните координати.\\

Clicks & Километри.\\
"Enemy at 12 o'clock high" & В авиацията ориентирането става като по положението на часовете в часовника... "12 часа" - право напред по курса, "3 часа" - право отдясно, "6 часа" - право отзад /откъм опашката/, "9 часа" - право отляво... "High/low" - противниковите самолети атакуват /налитат отгоре/отдолу по вертикала... В случая "12 o'clock high" означава "вражески самолети право по курса атакуват отгоре"... \\


\end{longtable}