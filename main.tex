\documentclass[12pt, oneside]{report}
\usepackage{amsmath}
\usepackage{amssymb}
\usepackage{amsthm}
\usepackage{graphicx}
\usepackage{color}
\usepackage{hyperref}
\usepackage{enumerate}

\usepackage{url}
\author{}

\setcounter{tocdepth}{1}

\usepackage{hyphenat}
\usepackage[T1, T2A]{fontenc}
\usepackage[utf8]{inputenc}
\usepackage[main=bulgarian, english, japanese]{babel}
\usepackage{CJKutf8}

\usepackage{array}
\usepackage{tabularx}
\usepackage{caption}
\usepackage{longtable}

\let\cleardoublepage\clearpage

% \usepackage{fontspec}
% \setmainfont{CMU Serif}
% \setsansfont{CMU Sans Serif}
% \setmonofont{CMU Typewriter Text}

% \usepackage{xeCJK}
% \setCJKmainfont{ipam.ttf}
% \setCJKsansfont{ipag.ttf}

\selectbglanguage
\setlocalecaption{bulgarian}{contents}{Съдържание}
\setlocalecaption{bulgarian}{chaptername}{Глава}
\usepackage[toc, page, titletoc,title]{appendix}
\usepackage{pdfpages}

\title{ТРУДОВ РЕЧНИК\\
\large{Японски и китайски имена, английски, военни и спортни термини}}

\begin{document}

\maketitle

\selectbglanguage

\section*{Благодарности}
    
Благодарности за изготвянето на речника: Ragnos и postmortem от Eastern Spirit, KoP3Le7o, DARK RANGER, Alexeev, nightwarrior, Chep\_92, Hristo Lishev, BestRipper, Stone, simo76, Анубис, Phoebes, Ice-Man, D300, elisiaelf, SonGoku, ludoto\_mimi, shtipliiska, danissimo, \-\=\ F\ o\ z\ z\ y\ \=\-, Zaza14, Николай, riburan, ex-cuckoo, kelly, spotbeam, pinko™, kaloo, rburan, fakelini, motleycrue, burndead, milenski, freakazoid, star6inkata, sed, tnt1918, mass\_effect, cheeseus, Скуби Ду, o6ina, Guilty, dkp

\quad

Информацията е събрана в периода 2013-2015 г.

\quad

Графично оформление от Rinto-kun.

\tableofcontents

\listoftables

\chapter{Японски имена}
\label{chap:imena}
\section{Основни правила при транскрипцията на японски имена}
\subsection{Дългите гласни}
В практическата транскрипция е прието \textbf{дължината на гласните} да не се предава.
Когато обаче йероглифният запис на \textbf{съответното име} или \textbf{термин} е даден в скобки, той може да бъде \textbf{транскрибиран на кирилица}, като се отрази и дължината на гласните.
Затова например глинените фигурки от \textit{периода Джьомон} в текста се срещат като \textit{догу}, но йероглифният запис е транскрибиран като \textit{догуу}.

Аналогичен е случаят и с термина \textit{шинто}, който всъщност звучи като \textit{шинтоо}.
При необходимост предаването на дългите гласни може да стане по няколко начина: чрез повторно изписване на съответната гласна; чрез чертица над буквата; чрез двоеточие след нея. Така дългата гласна \textbf{/a/} може да се срещне като \textbf{/aa, \-а, a:/}.
В англоезичната литература е разпространено и предаването на дължината с помощта на буквата h. Така например названието на \textit{японския театър Но}, което в действителност звучи \textbf{/ноо/},
на английски се транскрибира като \textbf{Noh}.

\subsection{Редукция на гласните}
Типична за японския език е редукцията на тесните гласни \textbf{/u, i/} в неударени срички и в края на думите, както е например в \textbf{копулата /десу/} и в \textbf{суфикса /масу/}.
Тъй като в тези случаи \textbf{редукцията} е почти пълна, при предаване на японски имена с наставка \textbf{суке}, е по-добре тя да се изписва като \textbf{ске}, т.е \textit{Дайске}, а не \textit{Дайсуке}.

\subsection{Двойните съгласни}
Двойните съгласни се предават при свързването на \textbf{две морфеми} и при т.нар. \textbf{експресивна геминация}. Името на страната може да бъде произнесено или като \textit{Нихон}, или като \textit{Ниппон}, но не и като \textit{Нипон}. Важно е да се знае, че \textit{Нихон} и \textit{Ниппон} не са напълно взаимозаменяеми.
Например в спортни призиви при международни мачове се използва \textit{„Ниппон – Ганбаре Ниппон“}!

В японския се среща\textbf{ удвояване на съгласните}, което в английската транскрипция се отбелязва
просто с \textbf{удвоение на буквите}, като например \textbf{"kk"}, \textbf{"tt"}. Това се прави и в българската транскрипция. Т.е. в такива случаи се транслитерира буквално - \textbf{tt = тт, kk = кк}, и т.н.

\subsection{Морообразуващата съгласна /н/}
В японски език се срещат \textbf{два вида съгласни /н/}. Едната не се различава от българската и в позиция пред букви, предаващи йотувани гласни, се смекчава. Другата, която се записва с буквата \begin{CJK*}{UTF8}{song}ん\end{CJK*}, е т.нар. \textbf{морообразуваща съгласна /н/}, която винаги се \textbf{произнася като} \textbf{твърда}. В английската транскрипция този звук се предава чрез \textbf{апостроф (n’),} както в заглавието на антологията \textbf{Man’y\-osh\-u}, ако е отразена дължината на гласните, и като \textbf{Man’yoshu}, ако не е отразена.
В руската транскрипция се използва буквата \textbf{\"e}, която се нарича \textbf{твердый знак} и при необходимост изпълнява ролята на \textbf{апостроф}, както например в \textbf{Манъ\"eсю}.
В българската азбука \textbf{ъ} има самостоятелна звукова стойност и не може да се използва като апостроф. Тъй като употребата на апостроф, макар и с по-други функции, се допуска от българския правопис, тук ще го използвам в съчетание с буквата \textbf{н} за предаване на морообразуващата съгласна, както например \textbf{Ман’йошю}.

\subsection{Меките ж, ч, ш}
Българският правоговор изисква съгласните \textbf{ж, ч, ш} да се произнасят като \textbf{твърди}, но в японския език те са \textbf{меки съгласни}. Сричката 
\begin{CJK*}{UTF8}{song}しゅ\end{CJK*} например се предава или като \textbf{шу} под влияние на английската транскрипция, или като \textbf{сю} под влияние на руската. По-близо до японското произношение обаче е \textbf{шю}. Ето защо името на най-големия японски остров например е \textbf{Хоншю}, а не \textbf{Хоншу} или \textbf{Хонсю}.

\subsection{Дзен или зен}
Под влияние на \textbf{английската транскрипция} често се пише \textbf{з} вместо \textbf{дз}, като дори се твърди, че това е правилното произношение. Ако обаче привържениците на това твърдение помолят някой японец да произнесе например думата \textbf{защо}, със сигурност ще чуят \textbf{дзащо}.

\subsection{Сричката /цу/}
Африкатата \textbf{/ц/,} която се среща в японската сричка \textbf{/цу/,} при транскрипция с латиница се предава с буквосъчетанието \textbf{/ts/.} Тъй като този звук не се различава от българския \textbf{/ц/,} \textbf{при} \textbf{транскрибирането} му не би трябвало да има колебания.
Но авторите, които \textbf{механично следват английската транскрипция}, често пишат или\textbf{ тс, }или\textbf{ }дори\textbf{ тц}. Така \textbf{Цукуба} се среща и като \textbf{Тсукуба}, и като \textbf{Тцукуба}. \\\\

\textbf{Източник:} „\textbf{\textit{Японската цивилизация}}“ от Братислав Иванов.
 
\begin{table}[htbp]
    \centering
    \includegraphics[width=\textwidth]{chapters/transcription-1.JPG}
    \includegraphics[width=\textwidth]{chapters/transcription-2.JPG}
    \caption{Транскрипционна таблица}
\end{table}

\clearpage

\section{Японски език за българи}
Следващата част е извадена от учебното пособие „Японски език за българи“ от Атанас Атанасов и Томоки Ватанабе.

\includepdf[pages={2-6},lastpage=24]{textbook.pdf} 
  

\chapter{Японски обръщения}
В тази част смятам да поясня малко повече за обръщенията, които се използват
в японския език и \textbf{как} и \textbf{дали} да ги \textbf{превеждаме}. Като начало трябва да се каже,
че ако се превеждат тези обръщения, никак не е уместно да се ползват множеството обръщения, навлезли от английски, френски и т.н. от сорта на "\textbf{сър}", "\textbf{мистър}", "\textbf{мадам}" и т.н.,
\textbf{освен ако контекстът изрично не го изисква}. Също трябва да спомена, че някои от обръщенията
са \textbf{непреводими} и ако държите да превеждате обръщенията, просто ще се наложи да ги пропускате.

\section{Често срещани}
\begin{longtable}{|p{0.2\textwidth}|p{0.8\textwidth}|}
\hline
\textbf{Наставка} & \textbf{Превод} \\ \hline
\endfirsthead
\hline
\endhead
\hline
\endfoot

\textbf{-san}& г-н, г-жа, г-жица \\
\textbf{-chan/-kun}& практически \textbf{непреводимо}, оставяте само името \\
\textbf{-sama/-dono}& господарю, господарке (понякога се използват с ироничен привкус и тогава може да използвате различни вариации като \textbf{,,всемогъщи''}, \textbf{,,вездесъщи''} и т.н., но \textbf{само в шеговит контекст}) \\
-\textbf{sempai}& може да се каже, че е непреводимо, евентуално може като \textbf{,,старши''}, но \textbf{само ако става въпрос за отношения на работното място}, например \\
\textbf{-kouhai}& по-неопитен и по-низш в йерархията, няма адекватен превод \\
\textbf{-sensei}& евентуално може да се преведе като \textbf{,,учителю''}, но съветвам и тук по-скоро да се оставя само името, тъй като \textbf{това обръщение се използва и за лекари, професори и т.н.} \\
\textbf{-senshuu}& за спортисти, евентуално като \textbf{,,състезател''}, или в зависимост от спорта, за който става въпрос. \\

\hline\caption{Основни обръщения в японския език}

\end{longtable}

\section{Фамилни имена}

\begin{table}[htbp]
    \centering
        \begin{tabular}{|p{0.15\textwidth}|p{0.85\textwidth}|}
            \hline
            \textbf{Наставка} & \textbf{Превод} \\ \hline
            \textbf{okaa-san / kaa-san}&майка, мамо\\ 
        \textbf{otou-san / tou-san}&татко, тате\\ 
        \textbf{onee-san / nee-san}&кака, како (както и в българския, може да се обръщат с "како" и към по-голямо момиче, което не е в семейството, но им е близко)\\ 
        \textbf{onii-san / nii-san}&батко, бате (казаното за \textbf{onee-san} важи и тук)\\ 
        \textbf{otouto}&по-малък брат, братче (тъй като в българския нямаме определена дума за това ви предлагам да не го превждате навсякъде където се споменава, а да го замествате с име или местоимение)\\ 
        \textbf{imouto}&по-малка сестра, сестричка (казаното за "otouto" важи и тук)\\ 
        \textbf{ojii-san / jii-san}&дядо (спокойно се използва и за хора извън семейството, както и в български)\\ 
        \textbf{obaa-san / baa-san}&баба, бабо (и това също се използва за хора извън семейството)\\ 
        \textbf{oji-san}&чичо\\ 
        \textbf{oba-san}&леля, лельо (и ,,чичо'', и ,,лельо'' пак са обръщения, които могат да се използват към хора, които не са роднини)\\ 
        
        \hline
        \end{tabular}
    \caption{Фамилни имена}
    \label{tab:familynames}
\end{table}

Както е видно, с изключение на \textbf{,,майка''} и \textbf{,,татко''}, всички други могат да бъде използвани и извън семейството.
Съветвам ви в тези случаи \textbf{не винаги да ги превеждате}, а да използвате \textbf{имена} и \textbf{местоимения} като \textbf{заместители}, където е възможно, защото японците не се уморяват да ги повтарят и \textbf{на български понякога звучи тромаво или странно.}


\section{Армейски звания}
Някои от по-често употребяваните звания в армията са описани в таблиците по-долу.
\begin{table}[htbp]
    \centering
    \begin{tabular}{|m{9em}|m{9em}|m{16em}|}
        \hline
        Японски & Английски & Български\\
        \hline
        \textbf{Nishi \begin{CJK*}{UTF8}{song}
            (二士)
        \end{CJK*}} & Private 2nd Class & \textbf{редник}\\ 
        \textbf{Isshi \begin{CJK*}{UTF8}{song}
            (1士)
        \end{CJK*}}& Private 1st Class & \textbf{редник}\\ 
        \textbf{Shichou \begin{CJK*}{UTF8}{song}
            (士長)
        \end{CJK*}} & Leading private & \textbf{ефрейтор}\\ 
        \textbf{Sansou \begin{CJK*}{UTF8}{song}
            (三曹)
        \end{CJK*}} & Sergeant & \textbf{младши сержант} (най-младши командир)\\ 
\textbf{Nisou \begin{CJK*}{UTF8}{song}
    (二曹)
\end{CJK*}} & Sergeant first-class & \textbf{сержант} (младши командир)\\ 
\textbf{Issou \begin{CJK*}{UTF8}{song}
    (一曹)
\end{CJK*}} & Master sergeant & \textbf{старши сержант} (старши командир)\\ 
\textbf{Souchou \begin{CJK*}{UTF8}{song}
    (曹長)
\end{CJK*}} & Master sergeant; sergeant major & \textbf{старшина} (най-старши командир) \\
\hline
    \end{tabular}
    \caption{Нисши звания в японските отбранителни сили – JSDF
    (сухопътни войски)}
\end{table}

\begin{table}[htbp]
    \centering
    \begin{tabular}{|m{9em}|m{9em}|m{16em}|}
        \hline
        Японски & Английски & Български\\
        \hline
        \textbf{Jōtōhei Kimmusha \begin{CJK*}{UTF8}{song}
            (上等兵勤務者)
        \end{CJK*}} & Acting Senior Private & \textbf{Редници}\\ 
        \textbf{Nitōhei \begin{CJK*}{UTF8}{song}
            (二等兵)
        \end{CJK*}}& Private 2nd Class  & \textbf{редник} (в съвременната бълг. армия няма редници от 2-ри и 3-ти клас)\\ 
        \textbf{Ittōhei \begin{CJK*}{UTF8}{song}
            (一等兵)
        \end{CJK*}} & Private 1st Class  & \textbf{редник}\\ 
        \textbf{Gochō Kimmu jōtōhei \begin{CJK*}{UTF8}{song}
            (伍長勤務上等兵)
        \end{CJK*}} & Junour Corporal & \textbf{Ефрейтори} \\
        \textbf{Jōtōhei \begin{CJK*}{UTF8}{song}
            (上等兵)
        \end{CJK*}} & Seniour Private & \textbf{ефрейтор} \\
        \textbf{Heichō  \begin{CJK*}{UTF8}{song} 
            (兵長)
        \end{CJK*}} & Lance Corporal & \textbf{ефрейтор}  \\       \textbf{Gochō \begin{CJK*}{UTF8}{song}
            (伍長)
        \end{CJK*}} & Corporal & \textbf{ефрейтор} \\
        \hline
    \end{tabular}
    \caption{Нисши звания в японската армия през ВСВ (сухопътни войски)}
\end{table}

\begin{table}[htbp]
    \centering
    \begin{tabular}{|m{9em}|m{9em}|m{16em}|}
        \hline
        Японски & Английски & Български\\
        \hline
        \textbf{Gunsō \begin{CJK*}{UTF8}{song}
            (軍曹)
        \end{CJK*}} & Sergeant & \textbf{Сержант}\\ 
        \textbf{Sōchō \begin{CJK*}{UTF8}{song}
            (曹長)
        \end{CJK*}}& Sergeant Major  & \textbf{Старшина} \\
        \hline
    \end{tabular}
    \caption[Сержантски звания]{Сержантски звания. В съвременната българска армия имаме младши сержант (най-младши командир); сержант (младши командир); старши сержант (старши командир); старшина (най-старши командир).}
\end{table}

\section*{Бележки}
\begin{itemize}
    \item \textbf{heichou} e нисше звание - редник или ефрейтор (ако е във флота - матрос или старши матрос). Най-общо войници.
    \item \textbf{taichou} е сержант или старшина, командващ взвод или рота.
    \item \textbf{buntaichou} е командир на екип или отряд в пожарната, обаче при нас пожарникарите май нямат звания и мисля, че еквивалент няма.
    \item \textbf{danchou} трябва да е лидер (на нещо си), упълномощено лице, водач - на група, на партия, на фирма и т.н.
    Доста широко понятие и не намерих да е свързано с нещо конкретно.
\end{itemize}



\include{chapters/pravopis.tex}
\chapter{Китайски термини}
\section{Таблица на китайските династии}
\href{https://bg.wikipedia.org/wiki/%D0%9A%D0%B8%D1%82%D0%B0%D0%B9%D1%81%D0%BA%D0%B8_%D0%B4%D0%B8%D0%BD%D0%B0%D1%81%D1%82%D0%B8%D0%B8}{Линк към Уикипедия.} 

\section{Транслитериране на китайските имена}

Таблицата е предоставена от \textbf{Алексей СКОБЛИКОВ}. \\
Българската транскрипция е редактирана от \textbf{Яна ШИШКОВА}.

\href{http://china.edax.org/?p=326}{Източник: Всичко за Китай.}

Достъпна в Таблица \ref{tab:kitaiski_imena}.
\include{chapters/sport.tex}
\chapter{Военни термини}

\section{Военни звания}
\begin{longtable}{|p{0.25\textwidth}|p{0.25\textwidth}|p{0.25\textwidth}|p{0.25\textwidth}|}
\hline
\textbf{US Army} & \textbf{Български аналог СВ и ВВС} & \textbf{US Navy} & \textbf{Български аналог ВМС} \\
\hline
\endhead
\hline
\endfoot
\hline
\caption{Военни звания} \label{tab:voenni-zvanija} \\
\endlastfoot
\hline
 Private & Редник & Seaman & Матрос
\\ Corporal & Ефрейтор & Petty Officer Third Class & Старши матрос
\\ Sergeant & Сержант & Petty Officer First Class & Старшина І степен
\\ First Sergeant & Старши сержант & Senior Chief Petty Officer  &Главен старшина
\\ Sergeant Major & Старшина & Master Chief Petty Officer & Мичман
\\ Warrant Officer & Офицерски кандидат && Офицерски кандидат  
\\ Second Lieutenant & Младши лейтенант & Sub Lieutenant & Младши лейтенант
\\ Lieutenant & Лейтенант & Ensign & Лейтенант
\\ First Lieutenant & Старши лейтенант & Lieutenant Junior Grade & Старши лейтенант
\\ Captain & Капитан & Lieutenant & Капитан-лейтенант
\\ Major & Майор & Lieutenant Commander & Капитан ІІІ ранг
\\ Lieutenant Colonel & Подполковник & Commander & Капитан ІІ ранг
\\ Colonel & Полковник & Captain & Капитан І ранг
\\ Brigadier General & Бригаден генерал & Commodore (Lower Half) & Комодор
\\ Major General & Генерал-майор & Rear Admiral (Upper Half) & Контраадмирал
\\ Lieutenant General & Генерал-лейтенант & Vice Admiral  &Вицеадмирал
\\ General & Генерал & Fleet Admiral & Адмирал
\\
\end{longtable}

Други:
\begin{itemize}
    \item AAM = Lieutenant Colonel and above /от подполковник и нагоре/
    \item ARCOM = Colonel and above /от полковник и нагоре/
    \item MSM = Major General and above /от генерал-майор и нагоре/
    \item LM = Lieutenant General and above /от генерал-лейтенант и нагоре/
    \item DSM = U.S. Army Chief of Staff /началник-щаб на американската армия/
    \item AWOL = Absent Without Official Leave - напуснал без разрешение
    \item POW = prisoner of war - именно военнопленник, а не военнозатворник, затворник от войната и подобни.
\end{itemize}

\section{Военни термини и фрази}

\begin{longtable}{|p{0.3\textwidth}|p{0.7\textwidth}|}
\hline
\textbf{Фраза} & \textbf{Превод} \\
\hline
\endhead
\hline
\endfoot
\hline
\caption{Често употребявани фрази} \label{tab:frazi} \\
\endlastfoot
To read & слушам радиопредаване (качество на приемане, останало в езика на радиотелеграфистите от времето на азбуката на Морз)\\ 
"Do you read me" & Как ме чуваш?\\ 
"Read you loud and clear" & Чувам те силно и ясно.\\ 
Copy & "Разбрах". "Копирането" идва от времената, когато радистите записвали съобщението на хартия\\ 

Roger & "Разбрах ви" или "Разбрано". Не особено уставен отговор. Означава, че предадената информация е приета и разбрана. НЕ означава съгласие за изпълнение (молба или заповед). От първата буква на думата "Received"("получено").\\ 

Acknowledged & "Прието". Подтвърждение за приемане на предаването. По-дълъг аналог на "Roger".\\ 

Wilco & "Ще бъде изпълнено", "Слушам". Заповедта е разбрана и ще бъде изпълнена. Съкращение от Will comply.\\ 

Over & "Приемам". В случай не означава "Край" /End/\\ 
Over and out & "Край на връзката".\\ 
"Zipper" & Потвърди получаването на радиосъобщението с две кратки натискания на бутона на микрофона.\\ 
"There are Indians on your way" & "Имаш противник пред себе си" /на известно разстояние/\dots Останало съобщение от времето на Дивия Запад, когато индианците са били враг на "демократите".\\ 
\end{longtable}

\begin{longtable}
{|p{0.3\textwidth}|p{0.7\textwidth}|}
\hline
\textbf{Фраза} & \textbf{Превод} \\
\hline
\endhead
\hline
\endfoot
\hline
\caption{Военен радиообмен} \label{tab:callouts} \\
\endlastfoot
\hline

Tally & "Виждам целта!".\\
No joy & "Целта не се вижда!" (именно "не се вижда", а не "Тук няма никаква цел". Усетете разликата!)\\

Bandit & Вражески самолет/въртолет.\\

Bogie & Самолет/вертолет с неизвестна принадлежност.\\

Visual & "Виждам". Визуален контакт със своя самолет/въртолет.\\

Blind & "Не виждам". Отсъствие на визуален контакт със свой самолет/въртолет.\\

Winchester & "Празен съм". Предаващият е изразходвал боекомплекта си.\\

Go wet/dry & "Аз съм мокър/сух". Самолет/въртолет пресича бреговата линия в посока на морето/сушата.\\

Sunrise & "Изгрев". Сигнал на екипажа, че започва да получава данни от външни източници (AWACS или наземен радар).\\

Midnight & "Полунощ". Сигнал на екипажа, че спира да получава данни от външни източници (AWACS или наземен радар).\\

Tumblweed & "Нищо не се вижда". Отсъствие на визуален/радарен контакт с който и да е. Запитване за допълнителна информация.\\

Splash & "Целта е унищожена" (за самолет); "Пряко попадение" (за удар по наземна цели).\\

Fox one & Изстрелване на ракета с радиолокационно самонасочване.\\

Fox two & Изстрелване на ракета с инфрачервено /топлинно/ самонасочване.\\

Fox three & Изстрелване на ракета с увеличен обсег на действие AMRAAM или Phoenix.\\

Fox four & Стрелец от бомбардировач имитира стрелба по цел.\\

Fox mike & FM радио\\

Hotel fox & HF радио\\

Uniform & UHF/AM радио\\

Victor & VHF/AM радио\\

Angels & Височина в хиляди футове /трябва да се превърне в съответните метри /приблизително/.\\

Bingo & Сигнал, че в самолета/въртолета е останало само гориво според първоначалния план за действие и завръщане /без да се посяга на резерва/.\\

Joker & Сигнал (след Bingo), че в самолета/въртолета е останала гориво на краен минимум /връщане с използване на резервата/ - както му казваме "да се върне на бензинови пaри"/.\\

Bullseye & Условна точка на местността, от която се отчитат относителните координати.\\

Clicks & Километри.\\
"Enemy at 12 o'clock high" & В авиацията ориентирането става като по положението на часовете в часовника... "12 часа" - право напред по курса, "3 часа" - право отдясно, "6 часа" - право отзад /откъм опашката/, "9 часа" - право отляво... "High/low" - противниковите самолети атакуват /налитат отгоре/отдолу по вертикала... В случая "12 o'clock high" означава "вражески самолети право по курса атакуват отгоре"... \\


\end{longtable}
\chapter{Идиоми}
\section{Списък с основни идиоми}
\begin{description}
    \item[To (not) have a leg to stand on] - мога/ не мога да докажа нещо; имам/ нямам доказателства.\textit{ The problem is, if you don't have a witness, you don't have a leg to stand on. \dots He was sure there was a mole and when you catch them, you should have a leg to stand on. (Убеден беше, че има шпионин и когато го заловиш, трябва да имаш доказателства.)}
    \item[For good] - доста често не означава за добро, а завинаги
    \item[It does not make sense] - Странно, но и този израз се бърка доста често. Не означава "Не прави смисъл" или "Не се създава усещане", а Няма логика / Нещо не е така, както изглежда
    \item[We are in the same boat] - Не означава "С теб сме в една лодка", а че сме в една и съща ситуация; трябва да вършим нещата заедно, защото имаме един и същ проблем
    \item[You are barking up the wrong treе] – Не означава "Лаеш по грешното дърво", а се използва в ситуации, когато се опитваш да постигнеш нещо, но начинът, по който се опитваш да го направиш, е погрешен; или напълно грешиш или не си разбрал/схванал нещо
    \item[You can not have the best of both worlds] - Не означава "Не можеш да имаш най-доброто от двата свята", а не можеш да получиш всичко, което желаеш / трябва да направиш избор
    \item[A bad egg] - Не означава "Развалени яйца", а човек, на когото не може да се вярва или се държи нечестно. Съвсем спокойно в зависимост от контекста може да се използват и типично нашенските изрази "От кофти тесто съм замесен" или "От лоша семка съм"
    \item[Try walking in my shoes] - Определено не означава "Опитай се да ходиш с моите обувки", а Опитай се да се поставиш на мое място
    \item[Know-it-all] - всезнайко
    \item[to be/fall head over heels] - влюбен съм
    \item[to cost an arm and a leg] - имам много висока цена; струвам адски скъпо
    \item[hit the road] - не е ритвам пътя, а потеглям, тръгвам на път, хващам пътя
    \item[take a flight] - 1. тръгвам си, отивам си; 2. тръгвам /отпътувам със самолет
    \item[in cold blood] - действие извършено с изключително хладнокръвие и жестокост
    \item[under the weather] - не се чувствам добре
    \item[I am so blue] - Не означава "Толкова съм син", а Тъжно ми е
    to drop (someone) a linе – пиша /изпращам бележка или писмо на някого
    \item[enough is enough] - стига толкова, изчерпи ми се търпението
    \item[Back/bet on the wrong horse] - Да поддържаш/ защитаваш неподходящия човек, макар че дори и буквално преведено на български смисълът не се губи.
    \item[I am in two minds] - Двоумя се / не мога да взема решение
    \item[I am all ears] - Слушам нещо с интерес/Целият съм в слух
    \item[Alter ego] - много близък прятел. Буквалният превод на латинската фраза е «другото ми Аз»
    \item[Beating a dead horse] - не означава да биеш умрял кон mosking.\item[gif] - изразът се използва когато някой се опитва да предизвика интерес и внимание, но без никакъв успех
    \item[Big cheese] - не е голямо сирене, а се използва за шеф, подобно на нашето "голяма клечка"
    \item[Bite the bullet] - "не е захапи куршума" или "гризни дървото", а да се изправиш пред нещо неприятно, което не можеш да избегнеш
    \item[Break a leg] - определено не е "счупи си крак", а се използва като пожелание за успех, късмет
    \item[Cold feet] - не е "студени крака", а по-скоро еквивалент на нашето "разтреперват ми се мартинките", т.е. плашиш се, губиш кураж да свъриш нещо
    \item[Cold turkey] - ако го срещнете във филм, в който са намесени наркозависими, определено не означава студено пуешко, а "внезапно прекъсване на приема на наркотици"
    \item[Cuckoo in the nest] - не е "кукувица в гнездо" , а проблем, който ескалира бързо
    \item[Dead air] - не е умрял/мъртъв въздух, а пълна тишина
    \item[Dead man walking] - не е зобми или ходещ мъртвец, а човек, попаднал в беда или опасна ситуация, на път е да загуби работата си, обществени позиции, семейството си и т.н.
    \item[Devil's advocate] - не е адвокат на Дявола, а някой, който спори, извърта аргументите и защитава позиции, в които не вярва, но го прави от любов към спора. Може да се използва и нашият израз "чете Евангелието като Дявола"
    \item[Dog days] - не означава "кучешки дни", а много горещи летни дни
    \item[Dog eat dog] - съперничество, конкуренция
    \item[Don't push my buttons!] - Не ме нервирай/дразни
    \item[Donkey's years] - не е магарешки години, а дълъг период от време
    \item[Piеce of cakе] – не е парче торта, както често го виждам преведено, а фасулска работа, лесна работа
    \item[Jackass] - не е задник, а по-скоро тъпанар, идиот, глупак и т.н.
    \item[badass] - има много значения, но определено не е "злогъз". Най-често се използва за самоуверен и силен мъж тип "мачо"
    \item[Fast and furious] - понякого може да изначава "бързи и яростни", но като идиом се използва в ситуации, когато събитията се развиват мълниеносно
    \item[Fishy] - не е рибешко, а по-скоро е еквивалент на българското "Работата мирише", т.е. има нещо подозрително
    \item[Fly on the wall] - като идиом не означава "муха на стената", а се използва за някой, който е видял или чул нещо, т.е. свидетел на събитието
    \item[Get your feet wet] - придобивам опит, първи стъпки в някакво начинание
    \item[Give me a hand] - един друг израз, който се бърка много често и се превежда като "подай ми ръка", а в действителност означава помогни ми
    \item[Go bananas] - не е "върви за банани", а изразява душевно състояние на силно вълнение, тревога или безпокойство
    \item[Go fry an egg] - не е отиди да си изпържиш яйце, а разкарай се, остави ме на мира
    \item[Gone fishing] - не е отиде за риба, а отново изразява душевно състояние на обърканост, човек, който не е наясно какво се случва около него
    \item[Hair of the dog] - не е нито кучешка козина, нито косата на кучето, а чисто и просто махмурлук
    \item[Have a go] - друг израз, който често се превежда като "Върви" или "Имаш разрешение",
    а всъщност означава да се опиташ да направиш нещо, дори и да мислиш, че нямаш големи изгледи за успех , т.е. пробвай се
    \item[Over your head] - Заел си се с нещо, което е извън възможностите ти
    \item[have the guts/balls] - няма нищо общо с черва и други атрибути 3.gif , а е стиска ми, имам кураж да направя нещо
    \item[Kick ass] - жестоко, върховно
    \item[Put the moves on someone] - опитвам се да прелъстя, свалям
    \item[Way to go] - браво
    \item[in the middle of nowhere] - затруднено положение, безизходица, някъде далеч (на края на света)
    \item[Over your head] - Заел си се с нещо, което е извън възможностите ти
    to be/\item[sit on the fence] - изчаквам, колебая се, чакам да видя накъде ще духне вятъра
    \item[rock the boat] - създавам смут
    \item[Low man on the totem pole] - човек току-що постъпил на работа - новак в службата или най-ниско в служебната йерархия
    \item[tire kicker] - термин използван от дилърите на коли за човек,
    който постоянно се отбива в салоните за коли, оглежда колите, "подритва гумите", но никога не купува
    \item[Two-A-days] - или още познат като "Hell week", e футболен термин, който се използва, когато един футболен отбор има по две тренировки на ден
    \item[break my balls] - ядосвам, дразня, подигравам се с някого, правя номер на някого
    \item[To beat around the bush] - говоря с недомлъвки, увъртам
    (to be)\item[ like a bull in a China shop] - на български може да се преведе като "слон в стъкларски магазин"
    \item[The writing on the wall] - предзнаменование, поличба, знамение за нечия съдба, участ, орис или нещастие.
    \item[to all intents and purposes] - предимно или всъщност. В зависимост от контекста може да се преведе и като както и да го погледнем.
    \item[Dressed to kill] - спокойно, човекът никого няма да убива, той е просто много добре облечен.
    \item[Good shit] - Добра работа, добре свършено, браво!
    \item[Аt the end of the day] - В крайна сметка, в края на краищата
    \item[Pulling 'someone's leg] - подиграваш се с някой, правиш си бъзик
    \item[state of the art] - най-доброто средство; последна дума в техниката или технологията; ултрамодерно
    \item[to call a spade a spade] - назовавам нещата с истинските им имена
    \item[to come off cheap] - отървавам се леко
    \item[to lay by for a rainy day] - бели пари за черни дни
    \item[by hook or by crook] - на всяка цена
    \item[blood is thicker than water] - кръвта вода не става
    \item[as ugly as sin] - грозен като смъртта
    it never rains,\item[but it pours] - злото никога не идва само
    \item[to go through thick and thin] - минавам през огън и лед
    \item[I have to go/I got to go] - освен че значи трабва да вървя, може да се използва в хиляди комбинации като "трябва да свърша работата", "ходи ми се до тоалетната" и т.н
    \item[Goodbye for good] - сбогом завинаги
    \item[I'll/I will take it from here] - не е буквално аз ще го взема от тук, а аз поемам от тук.
    \item[just married] - младоженци
    \item[team] - комбина; често се превежда неправилно като отбор.
    \item[To ride shotgun] - возя се отпред, на седалката до шофьора.
    \item[Call shotgun] - заплювам си да се возя отпред ( ако е в този контекст )
    Out with the old, in with the new означава - каквото било - било; да забравим миналото (и да гледаме в бъдещето)
    \item[To be in the zone] - не означава, че някой се "намира в зоната", а означава, че е във върховата си форма, в стихията си, в силата си или в победна серия.
    \item[To drop a bomb on someone] - не означава, че "мятате бомба някому отгоре", а означава, че го шокирате.
    \item[Go postal] - да се вбесиш, да се ядосаш
    \item[to be on one's ear] - пиян съм
    \item[to be on one's second wind] - отпочинал съм
    \item[blue twoes] - полицейски автомобил
    \item[peeler] - полицай
    \item[to have more money than cents] - пилея пари, харча пари неразумно
    \item[to be away in the head] - върша необмислени неща
    \item[to be the gipsy in the house] - 1. неудачник 2. човек, който не се облича добре
    \item[SKELETON in the cupboard или family SKELETON] - неприятна/позорна семейна тайна.
    \item[SKELETONS in the closet] - тайна от миналото на някого, която е на път да бъде разкрита.
    \item[born with a SILVER SPOON in one's mouth] - роден с късмет, предопределен да бъде богат.
    \item[to SHOOT the bull] - хваля се, преувеличавам.
    \item[there will be the DEUCE to pay] - ще ти излезе coлено/през носа.
    \item[to ring a BELL] - разг. извиквам спомени, сещам се, спомням си.
    \item[God gives short horns to a cursed cow] - На бодлива крава Господ рога не дава.
    \item[German virgin] - Две девятки при игра на покер(Texas hold 'em).
    \item[my SHIP comes home] - забогатявам.
    \item[my SHIP has sailed] - разминавам се с богатството.
    \item[BLOWER] - телефон, тех. компресор.
    \item[out on a LIMB] - неизгодно/опасно положение.
    \item[gone to his REWARD] - починал, отишъл си от тоя свят, на оня свят
    \item[Elbowing someone out of the way] - Изблъсквайки с лакти някой, който ти се е изпречил на пътя
    \item[My knight in shining armour] - моят принц или моят приказен принц
    \item[Five-by-five / Five by five / 5 by 5 / 5-by-5 /] - Разбирам те напълно добре.
    Take the Michael /\item[ take the Mickey] - бъзикам се с някого
    \item[Bob's your uncle] - И всичко ще е точно
    \item[Get two bites at the cherry] - Постъпвам, действам неуверено
    \item[To hit the sack] - отивам да си лягам, гушвам възглавницата
    \item[to be up for grabs] - ставам обществено достояние, съм общодостъпен, за всички
    \item[Back to porridge] - обратно към всекидневието
    \item[to put my foot in my mouth] - казвам нещо обидно или глупаво; казвам нещо, за което после съжалявам
    \item[to bring/let/take someone down a peg or two] - скастрям някого, поставям някого на мястото му, смачквам фасона на някого
    \item[And what have you / And what you have] - и прочие.
    \item[Second to none] - на висота; по-добро от всичко.
    \item[You are avocado in my chop salad] - Специален си (специална си) за мен
    Keep (one's)\item[ head] - запазвам самообладание
    Clean (one's)\item[ clock] - побеждавам убедително
    \item[My teeth are floating] - много ми се ходи по малка нужда
    \item[few sandwiches short of a picnic] - луд или откачен. Демек шашав по цялата глава.
    \item[one card shy of a full deck] - същото като горното.
    \item[to go banana] - полудявам, откачам.
    \item[to pull the trigger] - освен дословно, където е необходимо, се превежда и като "върша нещо с ясното съзнание, че може да завърши зле за мен" или чисто и просто "действам решително"
    \item[like a bat out of hell] - правя нещо много бързо.
    \item[road hard and put away wet] - Нещо е използвано много и без да се полагат грижи за него. Лафът идва от язденето на коне. Преуморил си коня от яздене, та чак се е изпотил и после не си го сресал и попил, преди да го прибереш в конюшнята.
    \item[What's for tea?/What do you want for tea?] - Какво има за вечеря?/Какво искаш за вечеря?
    \item[slam dunk] - супер забивка
    \item[call the shots] - контролирам нещата
    \item[kick a habit] - спирам да върша нещо, което преди ми е било навик
    \item[kick back] - не значи ритам назад, а значи - да си лежа и да не правя нищо.
    \item[Get my rocks off/Got his rocks off] - Правя си кефа веднъж, той си направи кефа веднъж
    \item[to play it by ear/play something by ear] - отгатвам по нюх, по интуиция
    Every now and then/\item[Every once in a while] - от време на време, понякога
    \item[to go all the way] - стигам до края (в секса)
    \item[taking speed] - взимане на амфетамини
    to have/\item[get someone by the short hairs] - държа някого във властта си, водя някого за носа
    \item[to be on pins and needles] - на тръни съм, изправен на нокти, седя като на тръни
    \item[if/when push comes to shove] - ако/като ножът опре до кокала
    \item[You are quite dry] - Доста си безчувствен, неемоционален, безпристрастен, хладнокръвен, ироничен, жаден.
    When it rains,\item[ it pours] - От трън, та на глог
    \item[Drive up the wall] - да дразниш, да ядосваш някого
    \item[to go for wool and come home shorn] - връщам се с подвита опашка или готината приказка "отивам като аслан, връщам се посран"
    \item[Here's to you / here's how] - Наздраве!; За ваше здраве!
    \item[Here's to Mary!] – За Мери!
    \item[Here's to our victory!] – За победата (ни)!, а не Тук е победата ни!
    \item[it is Greek to me/you] - за мен/теб това е напълно неразбираемо; за мен това е като (написано) на китайски, звучи ми като на китайски
    \item[there is no use crying over spilt milk] - роненето на сълзи с нищо не помага; не може да върнеш времето назад; няма да върнеш изгубеното
    \item[a babe in arms] - като малко дете
    a babe in the wood(s) – доверчив, неопитен, некадърен, непрактичен човек; като малко дете
    \item[crow's feet] - пачи крак; бръчки в крайчеца на очите, а не гарванови крака, както го видях веднъж.
    \item[the shit hit the fan] - започват големи неприятности
    
\end{description}

\section{Цветни идиоми}
Тясната връзка между цветовете и емоциите намира колоритен израз - и буквално, и преносно, в образната реч на идиомите, в които участват цветове. По-долу са дадени най-употребяваните изрази, обагрени в бяло, черно, червено, розово и жълто; а във втората част ще разгледаме сини, зелени, кафяви, сиви, сребристи и златисти идиоми, както и такива, без определен цвят. 

\subsection{Бели идиоми}
В британската култура, както и у нас, белият цвят символизира чистота и невинност.

\begin{description}
    \item[Flying Head/Mind] - Това определено в преносен смисъл би трябвало да значи: Отнесен,мечтател, фантазьор и другите такива синоними...
    \item[tell a white lie] - (изричам) невинна, безобидна лъжа;
    \item[We take(step) on the wrong foot] - Буквално съм го виждал този израз как ли не преведен, но принципно е: Не почнахме като хората или сгрешихме в началото ... 99\% става дума за отношения между хората
    \item[as white as a sheet] - (ставам) блед като платно;
    \item[as white as a ghost] - смъртно бледен от страх/болест/шок;
    \item[white elephant] - ненужна, безполезна (често със скъпа поддръжка) собственост/придобивка;
    \item[a white-collar worker] - 1.служител в офис (ср. blue-\item[collar] - работник във фабрика, човек ,който се занимава с физ. труд). 2. ръководство, управление
    \item[great white hope] - надежден спасител, спасител от беда, спасител от големи проблеми
    \item[white coffee] - кафе с мляко (заб. но не и white tea; вместо това - \textbf{tea with milk})
    \item[whiter than white] - невинен като ангел; чист като сълза, в 1смисъл честен, искрен;
    \item[white sale] - традиционна разпродажба на чаршафи, кърпи, покривки и пр. на намалени цени.
\end{description}

\subsection{Черни идиоми}
Черният цвят традиционо свързваме със смъртта, вещерството, злите сили, празнотата, а след Черния четвъртък през 1929 г. този цвят придоби значение и на загуба, застой, отчаяние.

\begin{description}
    \item[in the black] - 1.на печалба; 2. печеливш, успешен (подход, идея, фирма);
    \item[black eye] - насинено (от удар) око;
    \item[black look] - свиреп поглед;
    \item[black and blue] - със синини (от удари) по тялото;
    \item[pot calling (calls) the kettle black] - присмял се хърбел на щърбел;
    \item[black out] - 1. затъмнявам (като изгасям осветлението или покривам прозорците); също - информационно затъмнение; 2. губя съзнание, припадам 3. цензурирам;
    \item[in black and white] - 1.черно на бяло, написано 2. ясен, ясно разбираем;
    \item[black-and-white] - черно-бял (и прен.); като кон с капаци;

\end{description}

\subsection{Червени идиоми}
Най-емоционално зареденият цвят – червеното, има и най-богата символика; свързва се с гняв, страст, любов, отмъщение, огън, здраве, война, жестокост, опасност и пр.; за червенокосите хора се говори, че са с огнен темперамент и силен дух.

\begin{description}
    
\item[as red as a cherry = as red as a poppy] - ярко червен;
\item[as red as a ruby = as red as blood] - тъмно червен;
\item[in the red] - на червено, на загуба (ср. in the black – на печалба);
\item[a redneck] - 1. невежа; селяндур. 2. „мутра”;
\item[a red flag] - червена лампа, в см. сигнал за опасност/нередност;
\item[red-eye flight] - нощен полет (когато самолетът излита късно през нощта и каца много рано сутринта);
\item[to be shown the red card] - уволнен съм;
to catch someone red-\item[handed] - хващам някого на местопрестъплението;
\item[to see red] - обезумявам от ярост, пада ми пердето;
\item[red tape] - бюрокрация (също - забавяне поради тромави административни процедури);
\item[roll out the red carpet] - посрещам някого с големи почести, устройвам салтанати;
\item[paint the (old) town red] - гуляя, забавлявам се по барове и кръчми; удрям го на живот;
\item[a red-letter day] - празничен, неработен ден (отбелязан на календара в червено);
\item[to see the red light] - надушвам приближаваща опасност, светва ми червената лампичка;
\item[red-light district] - район с червени фенери, т.е. където има проституция;
\item[redcap (ам)] – носач на гара или летище;
\item[red herring] - нещо, което отвлича вниманието от основния проблем; подвеждаща информация.
\item[like waving a red flag in front of a bull] - върша нещо, което със сигурност ще вбеси някого; играя си с огъня;
\end{description}

\subsection{Розови идиоми}
Розовото е момичешкият, "сладкият" цвят; символизира радост, оптимизъм, красиви мечти.

\begin{description}
    \item[pink slip] - предизвестие, дадено от работодателя, за прекратяване на трудов договор;
    \item[see pink elephants] - фантазирам, въобразявам си;
    \item[pink-collar workers] - офис служителки;
    \item[in the pink] - в отлично здраве;
    tickled pink –страшно доволен, кефлия.
\end{description}


\subsection{Жълти идиоми}
Жълтият цвят има двузначна символика; от една страна той се асоциира със светлината, вярата, божественото, а от друга – с малодушието, предателството, завистта и злобата.
\begin{description}
    \item[a yellow streak (to have a yellow streak down one's back )] – малодушие (като черта на характера) yellow-bellied (разг., остар.) - бъзлив, страхлив;
    \item[yellow-livered] - страхлив, малодушен;
    \item[be yellow] - страхопъзльо съм;
    \item[yellow flag] - карантина;
    \item[yellow press/journalism (ам)] – жълта преса / журналистика;
    \item[yellow jacket] - 1.сънотворно лекарство (от : барбитурата Нембутал, продаван през 70-те год. в жълти капсули); 2.някои видове дрога.
    \item[jack-of-all-trades] - широкоскроен човек; който го бива за всичко;
    \item[gardenvariety] - обикновен, типичен;
    \item[a clean bill of health] - добро здравословно състояние; добра диагноза от лекар;
    \item[to keep out of someone's hair] - не създавам проблеми на някого, не притеснявам някого, не го дразня.
    \item[taken for a ride] - прекарвам някого; "прецаквам" някого
    \item[take a pot luck] - рискувам
    \item[give somebody your backing] - помагам на някого
    \item[as dead as a dodo] - нещо без бъдеще, почти невъзможно
    \item[like a lamb to the slaughter] - букв. като агне за клане; покорно
    \item[he/she wouldn't hurt a fly] - букв. не би наранил и муха; невинен
    \item[fish out of water] - букв. като риба на сухо; в безизходица; не на място
    \item[chicken out] - плаша се
    \item[bee line] - най-краткият път до нещо; възможно най-бързо
    \item[walk like a cat on a hot tin roof] - ходи несигурно; нестабилно 
\end{description}


\subsection{Сини идиоми}
Синьото символизира интелект, мъдрост, духовно начало,
вярност, постоянство, искрена любов, непорочност, вечност, спокойствие.

\begin{description}
    \item[to blue pencil something] - цензурирам, налагам цензура;
\item[a blue-eyed boy] - 1. протеже; фаворит, млад мъж, ползващ се с подкрепата
на висшестоящ и лансиран заради специални заслуги. 2. млад мъж, приветстван (неохотно) за успехите, които е постигнал;
\item[to look / feel blue] - изглеждам/чувствам се тъжен;
\item[blue in the face] - посинял от яд; силно развълнуван (до пръсване);
\item[once in a blue moon] - от дъжд на вятър;
\item[men/boys in blue] - полицаи (униформени);
\item[out of the blue = out of a clear blue sky] - изневиделица, като гръм от ясно небе;
\item[blue funk (to be in a ~)] – отчаян до смърт, (изпадам) в черна дупка;
\item[blue ribbon] - първа награда; екстра качество, „супер”;
\item[talk a blue streak] - не ми млъква устата, говоря много и бързо;
\item[a blue movie] - филм за възрастни (с еротични сцени);
\item[scream blue murder] - оревавам орталъка; вдигам врява; протестирам шумно. 
\end{description}


\subsection{Зелени идиоми}
Зеленият цвят има дуалистичната природа – свързваме го както с живота, растежа, плодородието, така и със завистта, злобата, измамата.

\begin{description}
    \item[to be green] - незрял; новак; неопитен; аджамия;
\item[give someone the green light] - давам позволение, разрешавам (някому да направи/започне нещо);
\item[grass is always greener on the other side] - чуждата кокошка патка изглежда;
\item[green belt] - зелени площи край града;
\item[green power] - разг. престиж и власт, произтичащи от пари;
\item[green-eyed monster] - ревност;
\item[green with envy] - умирам, пръскам се от завист;
\item[green thumb] - талант за градинарство;
\item[green around the gills] - жълт-зелен (от страх/болест);
\end{description}

\subsection{Кафяви идиоми}
Кафявото свързваме със Земята, есента, меланхолията, покоя, консервативността.

\begin{description}
     \item[to be browned off] - отегчен (съм) до смърт; до гуша ми е дошло;
    \item[brown out] - смущения в електрозахранването, когато токът ту спира, ту идва;
    \item[brown paper bag] - (жарг. на радиоводещите в градовете) – полицейска кола без обозначения;
    \item[as brown as a berry] - черен като циганин (от слънце);
\end{description}

\subsection{Сребристи идиоми}
Среброто (като цвят) е символ на Луната, женското начало, непорочността, чистотата.

\begin{description}
    \item[the silver screen] - киното (като изкуство);
\item[every cloud has a silver lining] - всяко зло за добро.
\item[brown study] - замислено, вглъбено състояние/настроение;
\item[to brown-nose] - подмазвам се, „четкам”.
\end{description}


\subsection{Златисти идиоми}
Златото (като цвят) е символ на Слънцето, мъжкото начало, духовното извисяване, божественото, силата, безсмъртието.

\begin{description}
     \item[a golden opportunity] - златна възможност, голям шанс;
    \item[a golden handshake] - парично обезщетение при пенсиониране, най-вече на служител с висок пост;
    \item[a golden boy] - златно момче, най-вече в спорта - за изключително надарен и можещ състезател.
\end{description}

\subsection{Неоцветени идиоми}

\begin{description}
    \item[a horse of another/a different colour (ам)]- (това е) друг въпрос, нещо различно;
\item[to change colour] - пребледнявам;
\item[give colour to] - потвърждавам, придавам достоверност;
\item[(to be) colourless] - безличен (съм), незабележим (за човек, който няма ярка индивидуалност);
\item[off-colour joke] - мръсен виц;
\item[with flying colours] - с голям успех;
\item[to paint in bright/dark colours] - представям, обрисувам нещо в ярки/мрачни краски;
\item[to show oneself in one's true colours] - разкривам истинската си същност;
\item[to see someone in his true colours] - виждам истинското лице на някого (едва сега разбирам що за човек е);
\item[a highly coloured report] - доклад, който съдържа силно преувеличеили предубедена информация. 
\end{description}

\section{Британски изрази и идиоми}

\begin{description}
    \item[Bees Knees] - не означава "коленете на пчелите", а невероятен, приказен, фантастичен.
    \item[Wanker;Tosser] - означават едно и също нещо:онанист или по-просто казано чек***ия.
    \item[On your bike] - по-учтива форма на "fuck off"
    \item[Nancy boy] - изнежен мъж, но също така се употребява и за гей
    \item[Brassed off] - дошло ти е до гуша
    \item[Give us a bell] - обади ми се
    \item[On the job] - означава работя усърдно, но също така,правя секс
    \item[Total pants] - пълен боклук,обикновено се употребява за предмети,не за хора
    \item[Reverse the charges] - да звъниш на някого по телефона за негова сметка

\end{description}

\section{Други изрази}
\href{https://www.usingenglish.com/reference/idioms/}{Голяма база с идиоми може да намерите на този линк.}
\appendix

\appendixname

\begin{longtable}{|l|l|l|}
    \caption{Транскрипционна таблица за китайски имена} \label{tab:kitaiski_imena} \\
    \hline
    \textbf{PinYin} & \textbf{Българска} & \textbf{Палладий} \\
    \hline
    \endfirsthead
    \hline
    \textbf{PinYin} & \textbf{Българска} & \textbf{Палладий} \\
    \hline
    \endhead
    \hline
    \endfoot
    \hline
    \endlastfoot
    \multicolumn{3}{|c|}{A} \\
    \hline
    a &а &a\\
    ai &aй &ай
\\    an &aн &ань
\\    ang &aн &ан
\\    ao &ao &ао
 \\
    \hline
    \multicolumn{3}{|c|}{B} \\
    \hline
    ba &бa &ба
\\    bai &бай &бай
\\    ban &бан &бань
\\    bang &бан &бан
\\    bao &бао &бао
\\    bei &бей &бэй
\\    ben &бън &бэнь
\\    beng &бън &бэн
\\    bi &би &би
\\    bian &биен &бянь
\\    biao &бяо &бяо
\\    bie &бие &бе
\\    
    bin &бин &бинь
\\    bing &бин &бин
\\    bo &бо &бо
\\    bu &бу &бу 
\\
    \hline
    \multicolumn{3}{|c|}{C} \\
    \hline
    ca &ца &ца
\\    cai &цай &цай
\\    can &цан &цань
\\    cang &цан &цан
\\    cao &цао &цао
\\    ce &цъ &цэ
\\    cen &цън &цэнь
\\    ceng &цън &цэн
\\    ci &цъ &цы
\\    cong &цун &цун
\\    cou &цоу &цоу
\\    cu &цу &цу
\\    cuan &цуан &цуань
\\    cui &цуей &цуй
\\    cun &цун &цунь
\\    cuo &цуо &цо
\\    cha &ча &ча
\\    chai &чай &чай
\\    chan &чан &чань
\\    chang &чан &чан
\\    chao &чао &чао
\\    che &чъ &чэ
\\    chen &чън &чэнь
\\    cheng &чън &чэн
\\    chi &чъ &чи
\\    chong &чун &чун
\\    chou &чоу &чоу
\\    chu &чу &чу
\\    chuai &чуай &чуай
\\    chuan &чуан &чуань
\\    chuang &чуан &чуан
\\    chui &чуей &чуй
\\    chun &чун &чунь
\\    chuo &чуо &чо
\\    
\hline

\multicolumn{3}{|c|}{D} \\
\hline
da &да &да
\\dai &дай &дай
\\dan &дан &дань
\\dang &дан &дан
\\dao &дао &дао
\\de &дъ &дэ
\\dei &дей &дэй
\\deng &дън &дэн
\\di &ди &ди
\\dia &дя &дя
\\dian &диен &дянь
\\diao &дяо &дяо
\\die &дие &де
\\ding &дин &дин
\\diu &диу &дю
\\dong &дун &дун
\\dou &доу &доу
\\du &ду &ду
\\duan &дуан &дуань
\\dui &дуей &дуй
\\dun &дун &дунь
\\duo &дуо &до
\\
\hline

\multicolumn{3}{|c|}{E} \\
\hline
e &ъ &э
\\ei &ей &эй
\\en &ън &энь
\\eng &ън &эн
\\er &ар &эр
\\

\hline

\multicolumn{3}{|c|}{F} \\ \hline
fa &фа &фа
\\fan &фан &фань
\\fang &фан &фан
\\fei &фей &фэй
\\fen &фън &фэнь
\\feng &фън &фэн
\\fo &фо &фо
\\fou &фоу &фоу
\\fu &фу &фу
\\
\hline

\multicolumn{3}{|c|}{G} \\ \hline

ga &га &га 
\\gai &гай &гай
\\gan &ган &гань
\\gang &ган &ган
\\gao &гао &гао
\\ge &гъ &гэ
\\gei &гей &гэй
\\gen &гън &гэнь
\\geng &гън &гэн
\\gong &гун &гун
\\gou &гоу &гоу
\\gu &гу &гу
\\gua &гуа &гуа
\\guai &гуай &гуай
\\guan &гуан &гуань
\\guang &гуан &гуан
\\gui &гуей &гуй
\\gun &гун &гунь
\\guo &гуо &го
\\
\hline

\multicolumn{3}{|c|}{H} \\ \hline
ha &ха &ха
\\hai &хай &хай
\\han &хан &хань
\\hang &хан &хан
\\hao &хао &хао
\\he &хъ &хэ
\\hei &хей &хэй
\\hen &хън &хэнь
\\heng &хън &хэн
\\hong &хун &хун
\\hou &хоу &хоу
\\hu &ху &ху
\\hua &хуа &хуа
\\huai &хуай &хуай
\\huan &хуан &хуань
\\huang &хуан &хуан
\\hui &хуей &хуй
\\hun &хун &хунь
\\huo &хуо &хо
\\

\hline

\multicolumn{3}{|c|}{J} \\ \hline
ji &дзи &цзи
\\jia &дзя &цзя
\\jian &дзиен &цзянь
\\jiang &дзян &цзян
\\jiao &дзяо &цзяо
\\jie &дзие &цзе
\\jin &дзин &цзинь
\\jing &дзин &цзин
\\jiong &дзюн &цзюн
\\jiu &дзиу &цзю
\\ju &дзю &цзюй
\\juan &дзюен &цзюань
\\jue &дзюе &цзюе
\\jun &дзюн &цзюнь
\\

\hline

\multicolumn{3}{|c|}{K} \\ \hline

ka &ка &ка
\\kai &кай &кай
\\kan &кан &кань
\\kang &кан &кан
\\kao &као &као
\\ke &къ &кэ
\\ken &кън &кэнь
\\keng &кън &кэн
\\kong &кун &кун
\\kou &коу &коу
\\ku &ку &ку
\\kua &куа &куа
\\kuai &куай &куай
\\kuan &куан &куань
\\kuang &куан &куан
\\kui &куей &куй
\\kun &кун &кунь
\\kuo &куо &ко
\\

\hline

\multicolumn{3}{|c|}{L} \\ \hline
la &ла &ла
\\lai &лай &лай
\\lan &лан &лань
\\lang &лан &лан
\\lao &лао &лао
\\le &лъ &лэ
\\lei &лей &лэй
\\leng &лън &лен
\\li &ли &ли
\\lia &ля &ля
\\lian &лиен &лянь
\\liang &лян &лян
\\liao &ляо &ляо
\\lie &лие &ле
\\lin &лин &линь
\\ling &лин &лин
\\lou &лоу &лоу
\\lu &лу &лу
\\lь &лю &люй
\\luan &луан &луань
\\lьe &люе &люэ
\\lun &лун &лунь
\\luo &луо &ло
\\

\hline

\multicolumn{3}{|c|}{M} \\ \hline

ma &ма &ма
\\mai &май &май
\\man &ман &мань
\\mang &ман &ман
\\mao &мао &мао
\\me &мъ &мэ
\\mei &мей &мэй
\\men &мън &мэнь
\\meng &мън &мэн
\\mi &ми &ми
\\mian &миен &мянь
\\miao &мяо &мяо
\\mie &мие &ме
\\min &мин &минь
\\ming &мин &мин
\\miu &миу &мю
\\mo &мо &мо
\\mou &моу &моу
\\mu &му &му
\\

\hline

\multicolumn{3}{|c|}{N} \\ \hline

na &на &на
\\nai &най &най
\\nan &нан &нань
\\nang &нан &нан
\\nao &нао &нао
\\ne &нъ &нэ
\\nei &ней &нэй
\\nen &нън &нэнь
\\neng &нън &нэн
\\ng &н &нг
\\ni &ни &ни
\\nian &ниен&ниен
\\niang &нян &нян
\\niao &няо &няо
\\nie &ние &не
\\nin &нин &нинь
\\ning &нин &нин
\\niu &ниу &ню
\\nong &нун &нун
\\nu &ну &ну
\\nь &ню &нюй
\\nuan &нуан &нуань
\\nьe &нюе &нюэ
\\nuo &нуо &но
\\

\hline

\multicolumn{3}{|c|}{O} \\ \hline

ou &оу &оу \\ 

\hline

\multicolumn{3}{|c|}{P} \\ \hline

pa &па &па \\
pai &пай &пай
\\pan &пан &пань
\\pang &пан &пан
\\pao &пао &пао
\\pei &пей &пэй
\\pen &пън &пэнь
\\peng &пън &пэн
\\pi &пи &пи
\\pian &пиен &пянь
\\piao &пяо &пяо
\\pie &пие &пе
\\pin &пин &пинь
\\ping &пин &пин
\\po &по &по
\\pou &поу &поу
\\pu &пу &пу
\\
\hline

\multicolumn{3}{|c|}{Q} \\ \hline

qi &ци &ци \\
qia &ця &ця
\\qian &циен &цянь
\\qiang &цян &цян
\\qiao &цяо &цяо
\\qie &цие &це
\\qin &цин &цинь
\\qing &цин &цин
\\qiong &циун &цюн
\\qiu &циу &цю
\\qu &цю &цюй
\\quan &цюен &цюань
\\que &цюе &цюэ
\\qun &цюн &цюнь
\\

\hline

\multicolumn{3}{|c|}{R} \\ \hline

ran &жан &жань
\\rang &жан &жан
\\rao &жао &жао
\\re &жъ &жэ
\\ren &жън &жэнь
\\reng &жън &жэн
\\ri &жъ &жи
\\rong &жун &жун
\\rou &жоу &жоу
\\ru &жу &жу
\\ruan &жуан &жуань
\\rui &жуей &жуй
\\run &жун &жунь
\\ruo &жуо &жо
\\

\hline

\multicolumn{3}{|c|}{S} \\ \hline

sa &са &са
\\sai &сай &сай
\\san &сан &сань
\\sang &сан &сан
\\sao &сао &сао
\\se &съ &сэ
\\sen &сън &сэнь
\\seng &сън &сэн
\\si &съ &сы
\\song &сун &сун
\\sou &соу &соу
\\su &су &су
\\suan &суан &суань
\\sui &суей &суй
\\sun &сун &сунь
\\suo &суо &со
\\sha &ша &ша
\\shai &шай &шай
\\shan &шан &шань
\\shang &шан &шан
\\shao &шао &шао
\\she &шъ &шэ
\\shei &шей &шэй
\\shen &шън &шэнь
\\sheng &шън &шэн
\\shi &шъ &ши
\\shou &шоу &шоу
\\shu &шу &шу
\\shua &шуа &шуа
\\shuai &шуай &шуай
\\shuan &шуан &шуань
\\shuang &шуан &шуан
\\shui &шуей &шуй
\\shun &шун &шунь
\\shuo &шуо &шо
\\

\hline

\multicolumn{3}{|c|}{T} \\ \hline

ta &та &та
\\tai &тай &тай
\\tan &тан &тань
\\tang &тан &тан
\\tao &тао &тао
\\te &тъ &тэ
\\ten &тън &тэнь
\\teng &тън &тэн
\\ti &ти &ти
\\tian &тиен &тянь
\\tiao &тяо &тяо
\\tie &тие &те
\\ting &тин &тин
\\tong &тун &тун
\\tou &тоу &тоу
\\tu &ту &ту
\\tuan &туан &туань
\\tui &туей &туй
\\tun &тун &тунь
\\tuo &туо &то
\\

\hline

\multicolumn{3}{|c|}{W} \\ \hline

wa &уа &ва
\\wai &уай &вай
\\wan &уан &вань
\\wang &уан &ван
\\wei &уей &вэй
\\wen &уън &вэнь
\\weng &уън &вэн
\\wo &уо &во
\\wu &у &у
\\

\hline

xi &си &си
\\xia &ся &ся
\\xian &сиен &сянь
\\xiang &сян &сян
\\xiao &сяо &сяо
\\xie &сиун &сюн
\\xiu &сиу &сю
\\xu &сю &сюй
\\xuan &сюен &сюань
\\xue &сюе &сюэ
\\xun &сюн &сюнь
\\

\hline

\multicolumn{3}{|c|}{Y} \\ \hline

ya &я &я
\\yan &йен &янь
\\yang &ян &ян
\\yao &яо &яо
\\ye &йе &е
\\yi &и &и
\\yin &ин &инь
\\ying &ин &ин
\\yong &юн &юн
\\you &йоу &ю
\\yu &ю &юй
\\yuan &юен &юань
\\yue &юе &юэ
\\yun &юн &юнь
\\

\hline

\multicolumn{3}{|c|}{Z} \\ \hline

za &дза &цза
\\zai &дзай &цзай
\\zan &дзан &цзань
\\zang &дзан &цзан
\\zao &дзао &цзао
\\ze &дзъ &цзэ
\\zei &дзей &цзэй
\\zen &дзън &цзэнь
\\zeng &дзън &цзэн
\\zi &дзъ &цзы
\\zong &дзун &цзун
\\zou &дзоу &цзоу
\\zu &дзу &цзу
\\zuan &дзуан &цзуань
\\zui &дзуей &цзуй
\\zun &дзун &цзунь
\\zuo &дзуо &цзо
\\zha &джа &чжа
\\zhai &джай &чжай
\\zhan &джан &чжань
\\zhang &джан &чжан
\\zhao &джао &чжао
\\zhe &джъ &чжэ
\\zhei &джей &чжэй
\\zhen &джън &чжэнь
\\zheng &джън &чжэн
\\zhi &джъ &чжи
\\zhong &джун &чжун
\\zhou &джоу &чжоу
\\zhu &джу &чжу
\\zhua &джуа &чжуа
\\zhuai &джуай &чжуай
\\zhuan &джуан &чжуань
\\zhuang &джуан &чжуан
\\zhui &джуей &чжуй
\\zhun &джун &чжунь
\\zhuo &джуо &чжо
\\

\hline

\end{longtable}

 
\end{document}